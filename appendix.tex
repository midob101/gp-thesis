\section{Appendix}

Hier können Sie Ihren Anhang definieren.

Achten Sie darauf, dass der Anhang in Ihrer \texttt{thesis.tex}
initial auskommentiert ist.
Der entsprechende Part befindet sich nahe dem Ende der Datei.
Entfernen Sie bei Bedarf die Kommentierung um den Anhang nutzen zu können.


\section{Implemented grammars}

The grammar definitions that are implemented in the \verb|gp-modifiable-ast| library.

\subsection{MiniJava}

\begin{lstlisting}[
  basicstyle=\tiny, %or \small or \footnotesize etc.
]
LANGUAGE_DEF
  name = minijava;
  single_line_comment_available = true;
  single_line_comment_style = //;
  multi_line_comment_available = true;
  multi_line_comment_style_start = /*;
  multi_line_comment_style_end = */;
  case_sensitive = true;
  grammar_start = S;

LEXER_RULES
  whitespace              =   regex(\s+);
  identifier              =   regex([a-zA-Z_][a-zA-Z_0-9]*);
  integer_literal         =   regex([0-9]+(\.[0-9]+)?);
  true_literal            =   "true";
  false_literal           =   "false";
  this_expression         =   regex(this(?!([a-zA-Z_0-9])));
  new_expression          =   regex(new(?!([a-zA-Z_0-9])));
  class_stmt              =   regex(class(?!([a-zA-Z_0-9])));
  public_stmt             =   regex(public(?!([a-zA-Z_0-9])));
  static_stmt             =   regex(static(?!([a-zA-Z_0-9])));
  return_stmt             =   regex(return(?!([a-zA-Z_0-9])));
  extents_stmt            =   regex(extends(?!([a-zA-Z_0-9])));
  void_type               =   regex(void(?!([a-zA-Z_0-9])));
  main_identifier         =   regex(main(?!([a-zA-Z_0-9])));
  string_type             =   regex(String(?!([a-zA-Z_0-9])));
  int_type                =   regex(int(?!([a-zA-Z_0-9])));
  boolean_type            =   regex(boolean(?!([a-zA-Z_0-9])));
  array_length_stmt       =   regex(length(?!([a-zA-Z_0-9])));
  not                     =   "!";
  bracket_open            =   "(";
  bracket_close           =   ")";
  curly_bracket_open      =   "{";
  curly_bracket_close     =   "}";
  square_bracket_open     =   "[";
  square_bracket_close    =   "]";
  dot                     =   ".";
  comma                   =   ",";
  semicolon               =   ";";
  if                      =   regex(if(?!([a-zA-Z_0-9])));
  else                    =   regex(else(?!([a-zA-Z_0-9])));
  while                   =   regex(while(?!([a-zA-Z_0-9])));
  assignment              =   "=";
  plus                    =   "+";
  minus                   =   "-";
  multiply                =   "*";
  less_then               =   "<";
  and                     =   "&&";
  or                      =   "||";
  println                 =   "System.out.println";
  single_line_comment     =   customMatcher(singleLineCommentMatcher);
  multi_line_comment      =   customMatcher(multiLineCommentMatcher);

HIDDEN_LEXER_RULES
  whitespace, single_line_comment, multi_line_comment;

PRODUCTIONS
  S                           ->  MAIN_CLASS CLASS_DECLARATIONS[inline];
  CLASS_DECLARATIONS[list]    ->  CLASS_DECLARATION CLASS_DECLARATIONS | EPSILON;
  CLASS_DECLARATION           ->  class_stmt[hidden] identifier[alias=className] curly_bracket_open[hidden] 
                                  VAR_DECLARATIONS[inline] METHOD_DECLARATIONS[inline] curly_bracket_close[hidden];
  CLASS_DECLARATION           ->  class_stmt[hidden] identifier[alias=className] extents_stmt[hidden] 
                                  identifier[alias=extendedClassName] curly_bracket_open[hidden] 
                                  VAR_DECLARATIONS[inline] METHOD_DECLARATIONS[inline] curly_bracket_close[hidden];
  MAIN_CLASS                  ->  class_stmt[hidden] identifier[alias=className] curly_bracket_open[hidden] 
                                  public_stmt static_stmt void_type main_identifier bracket_open[hidden] string_type 
                                  square_bracket_open square_bracket_close identifier[alias=varName] 
                                  bracket_close[hidden] curly_bracket_open[hidden] VAR_DECLARATIONS[inline] 
                                  STATEMENTS[inline] curly_bracket_close[hidden] curly_bracket_close[hidden];
  METHOD_DECLARATIONS[list]   ->  METHOD_DECLARATION METHOD_DECLARATIONS | EPSILON;
  METHOD_DECLARATION          ->  public_stmt TYPE identifier[alias=functionName] bracket_open[hidden] 
                                  FORMAL_PARAMETER_LIST[inline] bracket_close[hidden] curly_bracket_open[hidden] 
                                  VAR_DECLARATIONS[inline] STATEMENTS[inline] return_stmt EXPRESSION 
                                  semicolon[hidden] curly_bracket_close[hidden];
  FORMAL_PARAMETER_LIST[list] ->  FORMAL_PARAMETER_LIST comma[hidden] FORMAL_PARAMETER | FORMAL_PARAMETER | EPSILON;
  FORMAL_PARAMETER            ->  TYPE identifier[alias=parameterName];
  VAR_DECLARATIONS[list]      ->  VAR_DECLARATIONS VAR_DECLARATION | EPSILON;
  STATEMENTS[list]            ->  STATEMENT STATEMENTS | EPSILON;
  VAR_DECLARATION             ->  TYPE identifier[alias=varName] semicolon[hidden];
  STATEMENT                   ->  BLOCK | 
                                  ASSIGNMENT_STATEMENT | 
                                  ARRAY_ASSIGNMENT_STATEMENT | 
                                  IF_STATEMENT | 
                                  WHILE_STATEMENT | 
                                  PRINT_STATEMENT;
  BLOCK                       ->  curly_bracket_open[hidden] STATEMENTS[inline] curly_bracket_close[hidden];
  ASSIGNMENT_STATEMENT        ->  identifier[alias=varName] assignment[hidden] EXPRESSION semicolon[hidden];
  ARRAY_ASSIGNMENT_STATEMENT  ->  identifier[alias=varName] square_bracket_open[hidden] EXPRESSION 
                                  square_bracket_close[hidden] assignment[hidden] EXPRESSION semicolon[hidden];
  IF_STATEMENT                ->  if[hidden] bracket_open[hidden] EXPRESSION bracket_close[hidden] 
                                  STATEMENT[alias=then] else[hidden] STATEMENT[alias=else];
  WHILE_STATEMENT             ->  while[hidden] bracket_open[hidden] EXPRESSION bracket_close[hidden] STATEMENT;
  PRINT_STATEMENT             ->  println[hidden] bracket_open[hidden] EXPRESSION bracket_close[hidden] 
                                  semicolon[hidden];
  TYPE                        ->  ARRAY_TYPE | 
                                  BOOLEAN_TYPE | 
                                  INTEGER_TYPE | 
                                  identifier[alias=className];
  ARRAY_TYPE                  ->  int_type square_bracket_open[hidden] square_bracket_close[hidden];
  BOOLEAN_TYPE                ->  boolean_type;
  INTEGER_TYPE                ->  int_type;
  EXPRESSION                  ->  AND_EXPRESSION
                                  | COMPARE_EXPRESSION
                                  | PLUS_EXPRESSION
                                  | MINUS_EXPRESSION
                                  | TIMES_EXPRESSION
                                  | ARRAY_LOOKUP
                                  | ARRAY_LENGTH
                                  | MESSAGE_SEND
                                  | CLAUSE[inline];
  AND_EXPRESSION              ->  CLAUSE[inline,alias=left] and[hidden] CLAUSE[inline,alias=right];
  COMPARE_EXPRESSION          ->  PRIMARY_EXPRESSION[inline,alias=left] less_then[hidden] 
                                  PRIMARY_EXPRESSION[inline,alias=right];
  PLUS_EXPRESSION             ->  PRIMARY_EXPRESSION[inline,alias=left] plus[hidden] 
                                  PRIMARY_EXPRESSION[inline,alias=right];
  MINUS_EXPRESSION            ->  PRIMARY_EXPRESSION[inline,alias=left] minus[hidden] 
                                  PRIMARY_EXPRESSION[inline,alias=right];
  TIMES_EXPRESSION            ->  PRIMARY_EXPRESSION[inline,alias=left] multiply[hidden] 
                                  PRIMARY_EXPRESSION[inline,alias=right];
  ARRAY_LOOKUP                ->  PRIMARY_EXPRESSION[inline] square_bracket_open[hidden] PRIMARY_EXPRESSION[inline] square_bracket_close[hidden];
  ARRAY_LENGTH                ->  PRIMARY_EXPRESSION[inline] dot[hidden] array_length_stmt;
  MESSAGE_SEND                ->  PRIMARY_EXPRESSION[inline,alias=calledOn] dot[hidden] identifier[alias=functionName] bracket_open[hidden] bracket_close[hidden] |
                                  PRIMARY_EXPRESSION[inline,alias=calledOn] dot[hidden] identifier[alias=functionName] bracket_open[hidden] EXPRESSION_LIST bracket_close[hidden];
  EXPRESSION_LIST[list]       ->  EXPRESSION | EXPRESSION_LIST comma[hidden] EXPRESSION;
  CLAUSE                      ->  NOT_EXPRESSION | PRIMARY_EXPRESSION[inline];
  NOT_EXPRESSION              ->  not[hidden] CLAUSE[inline];
  PRIMARY_EXPRESSION          ->  integer_literal
                                  | true_literal
                                  | false_literal
                                  | identifier[alias=varName]
                                  | this_expression
                                  | ARRAY_ALLOCATION_EXPRESSION
                                  | ALLOCATION_EXPRESSION
                                  | BRACKET_EXPRESSION;
  ARRAY_ALLOCATION_EXPRESSION ->  new_expression[hidden] int_type square_bracket_open[hidden] EXPRESSION square_bracket_close[hidden];
  ALLOCATION_EXPRESSION       ->  new_expression[hidden] identifier[alias=className] bracket_open[hidden] bracket_close[hidden];
  BRACKET_EXPRESSION          ->  bracket_open[hidden] EXPRESSION bracket_close[hidden];
\end{lstlisting}

  \subsection{Unterabschnitte}
  Die Verwendung von Unterabschnitten im Anhang
  mittels \texttt{\textbackslash subsection}
  funktioniert ebenfalls!
