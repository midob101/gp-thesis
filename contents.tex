%%%%%%%%%%%%%%%%%%%%%%%%%%%%%%%%%%%%%%%%%%%%%%%%%%%%%%%%%%%%%%%%%%%%%%%%%%%%%%%%
% Diese Datei beinhaltet den eigentlichen Inhalt Ihrer Arbeit.
%
% Es bietet sich der Übersicht halber an, die einzelnen Abschnitte jeweils
% in eigene Dateien zu schreiben und mittels \input einzubinden.
% Eine mögliche Verzeichnisstruktur sähe entsprechend so aus:
%
%     thesis/
%     +- tex/
%     |  +- introduction.tex
%     |  +- motivation.tex
%     |  +- experiments.tex
%     |  |  ...
%     |  +- conclusion.tex
%     +- abstract.tex
%     +- contents.tex
%     +- thesis.tex
%%%%%%%%%%%%%%%%%%%%%%%%%%%%%%%%%%%%%%%%%%%%%%%%%%%%%%%%%%%%%%%%%%%%%%%%%%%%%%%%

\section{Introduction}

\subsection{Motivation}


\subsection{Tools used}

Es wurde der machine learning assistant von intellij verwendet, welcher standartmäßig aktiviert ist, zum schreiben vom Code. Es wurde DeeplWrite/o.Ä. verwendet, um den Text der Arbeit einer grammatik/rechtschreibprüfung zu unterziehen.

\section{Prerequisites}

\subsection{Regular Expressions and Grammars}

Kurze zusammenfassung von regulären ausdrücken und grammatiken

\subsection{Lexer}

Kurze zusammenfassung was ein lexer macht.

\subsection{Parser}

Kurze zusammenfassung was der parser macht.

\subsubsection{LR(1)-Parsing}

Hier nur auf die Tabelle eingehen. Die Erstellung der Tabelle ist für die Arbeit irrelevant, die wird nicht erweiterbar sein und die Generierung ist zu komplex um es sinnvoll in der Arbeit unterzubringen.

\subsubsection{Preprocessor Statements}

Hier kurz darauf eingehen, was preprocessor statements sind.

\section{Implementation}

\subsection{Extendability}

Hier darauf eingehen, in wie weit das Programm erweiterbar ist. Welche Möglichkeiten existieren in den Lexer/Parser einzugreifen?

\subsection{Grammar file syntax}

\subsection{CST-Generation}

Hier darauf eingehen, wie ein Concrete syntax tree vom Parser generiert wird

\subsection{AST-Generation}

Hier darauf eingehen, wie man von einem Concrete syntax tree zu dem abstract syntax tree kommt.

\section{Proof of concept}

\subsection{Java Subset}

Hier kurz darauf eingehen, für welche Sprache eine Grammatik definiert wurde.

\subsection{Automated Refactoring}

Hier Beispiel/Beispiele für automatisierte Refactorings anführen. Beispielsweise Funktionsnamen von underscore zu camelCase überführen.

\section{Conclusion}

Am Ende der Arbeit werden noch einmal die erreichten Ergebnisse
zusammengefasst und diskutiert.
