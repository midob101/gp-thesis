\subsection{Regular Expressions and Grammars}

Based on the paper of Naom Chomsky \cite{chomsky} a grammar $G$ consists of four components. 

\begin{enumerate}
    \item A finite set $N$ of nonterminal symbols.
    \item A finite set $\Sigma$ of terminal symbols.
    \item A finite set $P$ of production rules, each defined as $(\Sigma \cup N)^*N(\Sigma \cup N)^* \rightarrow (\Sigma \cup N)^*$, where $^*$ is the kleene closure operator \cite[page 15]{theoretical_comp_sci}\footnote{The kleene operator indicates, that an element out of the set can appear zero or more times.}.
    \item A start symbol $S \in N$.
\end{enumerate}

A grammar can be described as a set of definitions and rules, which construct a language $L$.

By limiting the production rules $P$, several classes can be defined based on the Chomsky hierarchy \cite[pages 512-513]{formal_languages_and_automata}.
The main classes used in this thesis are the set of regular languages and the set of context free languages.

A language is a regular language, if it can be constructed by a regular grammar.
A grammar is a regular grammar, if for all productions $p \to q$ in $P$ the following applies: $p \in N$ and $q \in \Sigma \cup \Sigma N$ \cite[pages 32]{theoretical_comp_sci}.

A language is a context free language, if it can be constructed by a context free grammar.
A grammar is context free, if for all productions $p \to q$ in $P$ the following applies: $p \in N$ \cite[page 37]{theoretical_comp_sci}.

Regular expressions create the class of regular languages. Therefore, they can construct any language a regular grammar is able to construct \cite[pages 25-32]{theoretical_comp_sci}.
Regular expressions can be defined in a short and compact way.
The following statement is a regular expression:

\begin{align*}
a^*ba^*b
\end{align*}

This would construct all words that have any amount of "a", followed by exactly one b, followed by any amount of "a", followed by exactly one b.

However, regular languages are not sufficient to describe the syntax of a programming language. Regular languages are not able to describe the language of
opening and closing brackets for example.
This language requires for any opening bracket "(" a closing bracket ")".
A proof, that the language $a^nb^n$ with $n\in\mathbb{N}$ is not regular can be found in \cite[page 30]{theoretical_comp_sci}.

To handle these situations context free grammars are required, they are able to parse more languages and are sufficient enough, to parse most programming languages. 

An example for a context free grammar:

Let $N = \{E\}, \Sigma = \{a, b\}, S = \{E\}$ and the productions defined as:

\begin{align*}
P = \{&E \to a E b,\\
&E \to a b E,\\
&E \to \epsilon\}
\end{align*}

$\epsilon$ is a special symbol, it refers to the empty word. That means, that $E$ can be derived to nothing. Without this, the grammar would have endless self recursion.

This grammar would be equivalent to the opening and closing bracket example. This will parse any string, which has for one opening $a$ exactly one closing $b$.
