
\section{Implemented Grammars}

There are two grammars implemented in \verb|gp-modifiable-ast| by default. 
The first grammar is MiniJava. The MiniJava language was defined by \cite{Appel2002-kleene}. 

The second grammar is an extension to MiniJava, which was defined for this thesis. This extended grammar allows for more operations in comparisons.
This grammar is only used to illustrate some possible refactorings and use cases of \verb|gp-modifiable-ast|.

\subsection{MiniJava}

The implemented grammar is based on the transformed grammar in backus naur form from \cite{minijava_bnf}.

The MiniJava grammar has been defined by \verb|gp-modifiable-ast|. 
This is a very limited subset of Java, but it represents common constructs found in modern programming languages.
It includes basic object oriented programming, simple conditional statements, while loops and basic arithmetic operations. 
However, the grammar is quite limited. For example, it does not allow any other comparison operator then \lstinline|<| and not more 
then one arithmetic operation without brackets. As these limitations make it hard, to create meaningfull refactorings, the syntax of MiniJava was extended for this thesis to allow for other operations.

\subsection{Extended MiniJava}

The extended MiniJava grammar includes a small change to the grammar.

The grammar allows the use of more comparison operators. MiniJava only allows \lstinline|<|.

Extended MiniJava includes the other comparison operators \lstinline|>|, \lstinline|>=|, \lstinline|<=| and \lstinline|==|.

Each of those comparison operators can be refactored to \lstinline|<|, as MiniJava only supports integers and not floating point values.

The purpose of extended MiniJava is to perform a source code transformation, which converts any source code written in extended MiniJava to the
the MiniJava syntax. These types of transformations are commonly used in practice, for example by Babel \cite{babeljs}, which is used to transform
modern JavaScript into old JavaScript syntax. This allows developers to write code using the latest JavaScript standard while still supporting old browsers,
that run the transformed code.