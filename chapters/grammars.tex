
\section{Implemented Grammars}

\subsection{MiniJava}

The MiniJava language was defined by \cite{Appel2002-kleene}. The implemented grammar is based on the transformed grammar in Backus Natus Form from \cite{minijava_bnf}.

A grammar for a java subset has been defined, known as MiniJava. This is a very limited subset of Java, but it represents common constructs found in modern programming languages.
It includes basic object oriented programming, simple conditional statements, while loops and basic arithmetic operations. However, the grammar is quite limited. For example, it does not
allow any other comparison operator then \verb|<| and not more then one arithmetic operation without brackets. As these limitations make it hard, to create meaningfull refactorings, we have extended
the syntax of MiniJava aswell to allow for other operations.

\subsection{Extended MiniJava Subset}

The extended MiniJava grammar includes a few major changes. The first one allows for more comparison operators. Instead of only \verb|<| also \verb|>, <=, >=, ==| can be used.
Another change is that \verb|return| is now a normal statemet. This means, that \verb|return| can appear anywhere in method declarations and not only anymore at the end of a method.

These changes will allow us to demonstrate two refactorings which can be done with this tool. 
