\subsection{Architecture}

The only dependencies used are JUnit and JUnit-jupiter \cite{junit} for testing purposes.

The implementation is split in several packages.

\begin{itemize}
\item config\_reader: responsible for parsing the language definition files
\item grammar: provides classes to store the parsed grammar production rules.
\item language\_definitions: provides an interface to receive various definitions, like the production and lexer rules of a given language.
\item lexer: contains the lexer process
\item logger: internal logger class
\item parser: contains the LR(1) parser implementation
\item selectors: contains basic selectors and interfaces to define custom ones
\item syntax\_tree: contains classes for the CST, AST and the CST to AST conversion
\end{itemize}

The typical workflow will be as followed:

\begin{enumerate}
\item Parse a language definition, receive the language definition classes
\item Start the lexer process for a source file, receive the token stream
\item Start the parser process with the token stream, receive the CST
\item Start the process to transform the CST to an AST, receive the AST
\item Query the AST with selectors, receive nodes of the AST
\item Modify found nodes
\item Receive text from the AST, overwrite source file with the modified data
\end{enumerate}
