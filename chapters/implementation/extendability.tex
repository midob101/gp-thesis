\subsection{Extendability}

To allow for possible extensibility, we allow for several types of customization

\subsubsection{Lexer}

As mentioned before, lexer definitions can contain custom matchers. 
These matchers are custom implementations that are referenced by a name in the grammar file. 
These classes are written by the user and registered by calling \lstinline|CustomMatcherRegistry.registerCustomMatcher|. 
A custom matcher must implement the \verb|ICustomMatcher| interface. 
A matcher gets a \verb|LexerContext| and returns \verb|null| or the matched string. 
Using the \verb|LexerContext| parameter, the matcher can receive the remaining input and check if the beginning of the remaining input can be tokenized by this matcher.

The following example is an implementation of a custom matcher, that would match single line comments.

\begin{lstlisting}[language=Java, caption=Example of a custom matcher]
public class ExampleCustomMatcher implements ICustomMatcher {
    @Override
    public String match(LexerContext context) {
        if(context.getNextNChars(2).equals("//")) {
            // Find the end of line. This will be the comment token
            String remaining = context.getRemainingSource();
            Pattern p = Pattern.compile(".*(\r\n|\r|\n|$)");
            Matcher matcher = p.matcher(remaining);
            if (matcher.find()) {
                return matcher.group();
            }
        }

        return null;
    }
}
\end{lstlisting}

\subsubsection{AST Generation}

We allow several ways to customize AST generation. 
First, you can implement custom modifiers for grammar symbols.
You can also post-process and decorate the AST. 
For example, you can replace the tree nodes with your own implementation by extending \verb|AbstractSyntaxTreeNode|.

The following example is an implementation of a tree node that is simply a string.
This tree node can be used to add or replace code in the AST.

\begin{lstlisting}[language=Java, caption=Implementation of StringTreeNode]
public class StringTreeNode extends AbstractSyntaxTreeNode {
    private final String value;

    public StringTreeNode(String value) {
        this.value = value;
    }

    public String getDisplayValue() {
        return "StringTreeNode, value: " + this.value;
    }

    public String getValue() {
        return value;
    }

    protected String getSources() {
        return this.value;
    }
}
\end{lstlisting}

\subsubsection{Selectors}

Custom selectors can be defined by extending \verb|BaseSelector|. 
These can be easily defined and used with the query methods of the tree node.

The following example shows the implementation of a selector that matches a tree node based on the production used to create that node.

\begin{lstlisting}[language=Java,caption=Custom selector implementation]
public class ProductionSelector extends BaseSelector {
    private final String production;

    public ProductionSelector(String production) {
        this.production = production;
    }

    @Override
    public boolean matches(AbstractSyntaxTreeNode treeNode) {
        if(treeNode instanceof ProductionTreeNode convertedNode) {
            return convertedNode.getProduction()
                                .leftHandSymbol()
                                .name()
                                .equals(this.production);
        }
        return false;
    }
}
\end{lstlisting}
