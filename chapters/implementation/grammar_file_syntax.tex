\subsection{Grammar file syntax}
The productions and lexer definitions are defined separately from the implementations in their own text files. 
Each language has exactly one text file defining the productions and lexer definitions. The contents are divided into four different chapters.

The following example shows a grammar for arithmetic expressions that could be found in a programming language grammar.

This grammar allows to parse simple arithmetic expressions and allows comments and whitespaces.

\begin{lstlisting}[caption=Grammar file example]
LANGUAGE_DEF
    grammar_start = S;

LEXER_RULES
    add = "+";
    subtract = "-";
    multiply = "*";
    divide = "/";
    bracket_open = "(";
    bracket_close = ")";
    integer = regex(\d+);
    whitespace = regex(\s+);
    single_line_comment = customMatcher(singleLineCommentMatcher);
    multi_line_comment = customMatcher(multiLineCommentMatcher);

HIDDEN_LEXER_RULES
    whitespace, single_line_comment, multi_line_comment;

PRODUCTIONS
    S   ->  S[alias=left] add T[alias=right] | 
            S[alias=left] subtract T[alias=right] | 
            T;
    T   ->  T[alias=left] multiply F[alias=right] | 
            T[alias=left] divide F[alias=right] | 
            F;
    F   ->  bracket_open S bracket_close | 
            integer;
\end{lstlisting}

This grammar file would be able to parse the following sample code:

\begin{lstlisting}[language=Java, caption=Grammar file example]
// Arithmetic grammar test
(5+10) /** inline comment */ 
* (15*20)
\end{lstlisting}

The first chapter is \verb|LANGUAGE_DEF|. This contains some general definitions for the language. 
This chapter contains e.g. which comment styles are allowed and how they are identified, the name of the language, the file extension used, whether the language is case-sensitive, and which production should be used as the starting production, 
whether the language is case-sensitive or not, and which production should be used as the starting production. 
All settings here are defined as key-value pairs.

The second chapter is the \verb|LEXER_RULES|. 
This defines all the tokens that can occur in the source file and that the lexer should handle. 
The lexer can handle three different types of definitions. 
The first type matches a fixed string. The second type matches a regex. 
The last type uses a custom implementation to check for a match. 
The last matching definition wins. 
For example, the token definition \verb|true_literal = "true";| and the definition \verb|identifier = regex([a-zA-Z_][a-zA-Z_0-9]*);| would both match the string \verb|true|. 
Therefore, the more general \verb|identifier| token should be defined before the \verb|true_literal| token. 
This will cause the lexer to recognize the text \verb|true| as \verb|true_literal|. 
A lexer key must be written in all lowercase letters and underscore.

The chapter \verb|HIDDEN_LEXER_RULES| is a comma separated list of lexer definition keys, 
that should not be handled in the parsing process and should not be visible in the AST by default.
This category should mainly be used for whitespaces and comments. If these are not added in category, the grammar rules need to handle those.
As whitespaces and comments can be added basically everywhere, this would result in a very verbose definition of the productions.

The chapter \verb|GRAMMAR_RULES| contains all productions of this language. The syntax is similar, but not identical, to the backus natus form. (TODO: Source). The syntax has been extended with most of the modifiers suggested by \cite{GeneratingRewritableAST}. The terminals are the keys of the lexer definitions. The nonterminals are defined by the left sides of the productions. All nonterminals must be written in uppercase and underscore. \verb|EPSILON| is used for the empty word. Each terminal and nonterminal symbol can be extended by a list of modifiers in square brackets after the name. 
\verb|S -> S[alias=left] add T[alias=right]| defines the production with \verb|S| on the left and \verb|S add T| on the right. 
\verb|S| refers to the same production, \verb|T| refers to a different production. 
\verb|add| refers to a lexer definition. 
\verb|[alias=right]| sets an alias in the AST for the \verb|T| production.