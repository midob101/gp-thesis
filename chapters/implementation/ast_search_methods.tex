
\subsection{AST search methods}

Before any changes can be made to the AST, the nodes that need to be changed must first be found.

For this approach, each AST node has three methods.

\begin{enumerate}
\item \verb|query: Selector -> QueryResult|
\item \verb|queryChildren: Selector -> QueryResult|
\item \verb|queryImmediateChildren: Selector -> QueryResult|
\end{enumerate}

The \verb|query| method takes a selector and returns a \verb|QueryResult|.
The \verb|QueryResult| contains all nodes in the subtree of the searched node that match the \verb|selector|. 
The \verb|QueryResult| may also contain the searched node itself if it matches the \verb|selector|.

The \verb|queryChildren| behaves the same way, but does not include the searched node itself.

The \verb|queryImmediateChildren| will return only the immediate children of the searched node that match the \verb|selector|.

These methods are able to find hidden nodes.

The \verb|QueryResult| instance also allows to perform queries on the result. 
This will perform the corresponding method on all nodes in the result and create a new \verb|QueryResult| instance containing the merged results of each node. 
This allows for easy chaining of selectors. 
You can also merge \verb|QueryResult| instances for further processing with the \verb|merge| function.
