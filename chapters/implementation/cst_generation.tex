\subsection{CST-Generation}
The concrete syntax tree (CST) is a tree structure that exactly matches the grammar and lexer definitions. 
The generated CST should also contain nodes that are usually omitted, such as comments or whitespace.

The CST is implemented by a \verb|ConcreteSyntaxTreeNode| class. 
A node can reference either a production or a lexer token. 
We keep the productions, as they allow us to make advanced modifications and generate a less verbose abstract syntax tree from the CST. 
The lexer tokens are maintained because they contain the original sources.

This structure is generated directly by the parser. 
The parser maintains a second stack containing tree nodes that are not bound to a parent node.

As the parser passes through the token stream provided by the lexer, 
it generates new tree nodes and pushes them onto the stack whenever either the token is in \verb|HIDDEN_LEXER_RULES| or the action performed is a shift action. 

On a reduce action the parser will also create a new tree node. 
This tree node will have other tree nodes from the top of the stack as children. 
This behavior is very similar to how the state stack of the parser manages its state. 
The main difference is the tokens in the \verb|HIDDEN_LEXER_RULES|. 
The reduced production may have only one symbol on the right side, but on the tree node stack there are two nodes from \verb|HIDDEN_LEXER_RULES|. 
In this case the parser will pop three states from the top of the tree node stack and add them as children of the newly created node in reverse order. 
However, if the parser reaches the \verb|acc| state, it is not guaranteed that the stack contains only the root node. 
There may be tokens from \verb|HIDDEN_LEXER_RULES| which appeared at the beginning of the parsed string. 
So the parser will find the root node and prepend all other entries in the tree node stack to the children of the root node. 

With this algorithm, every single token from the lexer's token list is transformed into a tree node and placed in the CST. 
This guarantees that the whole source code is represented in the CST and that the CST could be transformed back to the source code exactly.

The following code shows a simplified version of the CST generation during the parser process.
\begin{lstlisting}[language=Java, caption=CST Generation]
if(isHiddenLexerRule() || isShiftAction()) {
    danglingTreeNodeStack.push(new TreeNode(currentParseAction));
} 

if(isReduceAction()) {
    TreeNode newTopOfStack = new TreeNode(currentParseAction);
    int i = getRightHandSymbolCount();
    while(i > 0) {
        TreeNode currentTopOfStack = danglingTreeNodeStack.pop();
        newTopOfStack.pushChild(currentTopOfStack);
        if(!currentTopOfStack.isHiddenLexerRule()) {
            i--;
        }
    }
    danglingTreeNodeStack.push(newTopOfStack);
}

if(isAcceptAction()) {
    TreeNode rootNode = danglingTreeNodeStack.pop();

    while(!danglingTreeNodeStack.isEmpty()) {
        TreeNode treeNode = danglingTreeNodeStack.pop();
        rootNode.prependChild(treeNode);
    }

    return rootNode;
}
\end{lstlisting}

