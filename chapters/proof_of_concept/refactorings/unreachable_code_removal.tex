\subsection{Unreachable code removal}

There are many cases of unreachable code, we will focus on just one case. Code that appears after a return statement in the same block, is code that will never be executed.
With this refactoring, we want to remove all code, which is happening after a return statement.

This is done by creating a selector to grab all \verb|return| statements. Once we done that, we receive the parent node of the \verb|return|. Now we loop over the children, and after we found the \verb|return| node, we begin to remove following nodes.

In this case, we will encounter an issue with maintaining the file formatting, as there may be multiple hidden nodes after the return statement.
For the purpose in extended minijava, we keep the last whitespace token, as this token is used between the last visible node in the AST and the end of the block.