
\subsection{Translation system}

This is an example for a refactoring, which may be required to be performed on a large scale and that is not doable by common tools.

Let a translation system be a system, which stores some data for some keys. This data includes the translations for each desired language and parameters.
A parameter is a variable, which gets dynamically inserted into the text. One of the parameters can be used as a pluralization parameter.

For example, the key \verb|"apple"| resolves to the english translation \verb|"{NAME} has {COUNT} apple"| if the parameter \lstinline|COUNT| is 1 
and \verb|"{NAME} has {COUNT} apples"| if \lstinline|COUNT| is not 1.
\lstinline|{NAME}| is a second parameter, which does not require any pluralization.

The wanted translation is received by calling \lstinline|translate("apple", {NAME: user, COUNT: i})|.


When switching to another translation system, which requires the pluralization parameter to be passed seperately, an issue occures. 
The desired call syntax will be: \lstinline|translate("apple", {NAME: user}, {COUNT: i})|

The old code does not include any information, which of the parameters is the pluralization parameter. 
We require the connection to an external source to correctly identify the wanted parameter.

This refactoring might be possible with regular expressions, but it would require a separate regular expression for each translation token.
Even if such a regular expressions were generated for each translation token, no regular expression will be able to parse all possible call syntaxes in most programming languages.
Therefore, not all replacements can be done by regular expressions, and only part of the refactoring will be done.
In the worst case, a regular expression might match unrelated code and introduce new problems.

The refactoring could be done manually, but this will be very time consuming if there are many translations.
Another rather simple solution would be to implement the adapter pattern \cite[pages 47-52]{java_design_patterns}. This design pattern allows us to use the old call syntax and the adapter translates it to the new call syntax at runtime.
However, this solution would leave us with outdated code.
The final option is to implement a custom solution.

With \verb|gp-modifiable-ast|, the sources can be parsed given a grammar, receive an AST, and identify the translation key and any parameters passed.
To identify which parameter is the pluralization parameter, the definition of the translation can be received. 
Then the pluralization parameter can be removed from the call syntax and added as a new parameter to the function call.

This refactoring has not yet been implemented, as both the MiniJava and extended MiniJava grammars are not sufficient to support an illustrative example.