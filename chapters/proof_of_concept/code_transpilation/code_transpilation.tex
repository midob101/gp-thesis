
\subsection{Code transpilation}

This project may also be used to transpile code from a source grammar to a target grammar. As an example, a transpiler was written, which transpiles applications that use
the extended MiniJava grammar to applications that use the regular MiniJava grammar. An example of a transpiler which performs operations like these would be babel. This tool is
used to transpile a newer version of JavaScript into an older one, which can be executed by more browser. That allows the developer to utilize new functionalities of the programming language while not breaking backwards compability.

The goal of this implementation is to show the power of a rewritable abstract syntax tree. Especially for extended return syntax, we have to apply severe modifications to the sources
to make it MiniJava compatible.

\subsubsection{Conditionals}

As MiniJava only supports the less then operator, we need to transform every conditional that uses a different operator.
For this, we use the following equivalent statements for $a, b \in \mathbb{Z}$:

\begin{align}
    a > b  &\Longleftrightarrow b < a\\
    a \leq b  &\Longleftrightarrow a < b + 1\\
    a \geq b  &\Longleftrightarrow b \leq a \Longleftrightarrow b < a + 1\\
    a = b  &\Longleftrightarrow (a \leq b) \land (a \geq b) \Longleftrightarrow (a < b + 1) \land (b < a+ 1)
\end{align}

As we have reduced all new comparison operators to the less then operator now, we can implement the refactoring.

This is done by creating a selector to find all comparisons, reading the comparison operator and performing the refactoring.