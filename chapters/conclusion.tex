\section{Conclusion and further work}

The implemented library is capable of generating an AST that preserves all information about the original sources.
This is done by parsing a language definition file, building a LR(1) parser, parsing the source code, and building an AST
which still contains all the tokens that are usually ignored by parsers.
With the provided API, it is possible to traverse and modify the AST in a way to implement custom modifications to a given source.

The grammar for MiniJava has been defined and several refactorings have been implemented to show the functionality and usability of the implementation.

A limiting factor is the effort required to define a language grammar. This could be improved by extending the grammar's
syntax to accept the extended backus natus form.

Another problem, due to the fact that the library is not implemented for a specific programming language, is the lack of 
functionality of complete compiler frontends, including but not limited to symbol tables. If there is a way to 
to create symbol tables and track the usage of certain variables, this library could be used for more refactorings.
Currently, these functions must be built on top of the library.

Therefore, this library is currently best used for refactorings to single nodes of an AST that contain all the information needed to perform the refactoring.