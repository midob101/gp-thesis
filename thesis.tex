%%%%%%%%%%%%%%%%%%%%%%%%%%%%%%%%%%%%%%%%%%%%%%%%%%%%%%%%%%%%%%%%%%%%%%%%%%%%%%%%
% Universität Düsseldorf                                                       %
% Lehrstuhl für Softwaretechnik und Programmiersprachen                        %
% Vorlage für Bachelor- und Masterarbeiten                                     %
% Erstellt: 2019-09-03                                                         %
%%%%%%%%%%%%%%%%%%%%%%%%%%%%%%%%%%%%%%%%%%%%%%%%%%%%%%%%%%%%%%%%%%%%%%%%%%%%%%%%
\documentclass{hhuthesis}


%%%%%%%%%%%%%%%%%%%%%%%%%%%%%%%%%%%%%%%%%%%%%%%%%%%%%%%%%%%%%%%%%%%%%%%%%%%%%%%%
%% Einstellungen zur Personalisierung                                         %%
%%                                                                            %%
%% Im Folgenden können Sie Ihre Arbeit personalisieren.                       %%
%%%%%%%%%%%%%%%%%%%%%%%%%%%%%%%%%%%%%%%%%%%%%%%%%%%%%%%%%%%%%%%%%%%%%%%%%%%%%%%%

%% Spracheinstellung
%% Kommentieren Sie die entsprechende Zeile ein bzw. aus.
%% Wir empfehlen jedem sich an einer englischen Arbeit zu versuchen.
\usepackage[ngerman,english]{babel} % English
%\usepackage[english,ngerman]{babel} % Deutsch

%% Ihr Name
\author{Michael Doberstein}

%% Der Titel der Arbeit
\title{A general-purpose modifiable AST parser}

%% Der zu erreichende Abschluss, entweder Bachelor oder Master
\gratuationtype{Bachelor}
% \gratuationtype{Master}

%% Beginn- und Abgabedaten der Arbeit
\begindate{29th of September 2024} % Beginn
\duedate{30th of December 2024} % Abgabe

%% Erst- und Zweitgutachter
\firstexaminer{Dr.~John~Witulski}
\secondexaminer{Dr.~Jens~Bendisposto}

%% Farb- oder Schwarzweißdruck
% Benutzen Sie das Kommando \blackwhiteprint,
% wenn sie in schwarzweiß drucken möchten.
% Im Farbdruck ist jede farbige Seite idR teurer.
% \blackwhiteprint  % Kommentarzeichen entfernen für Schwarzweißdruck

%%%%%%%%%%%%%%%%%%%%%%%%%%%%%%%%%%%%%%%%%%%%%%%%%%%%%%%%%%%%%%%%%%%%%%%%%%%%%%%%
%% (Ende) Einstellungen zur Personalisierung                                  %%
%%%%%%%%%%%%%%%%%%%%%%%%%%%%%%%%%%%%%%%%%%%%%%%%%%%%%%%%%%%%%%%%%%%%%%%%%%%%%%%%
%% LaTeX Packages in Nutzung                                                  %%
%%                                                                            %%
%% Im folgenden können Sie für die Niederschrift Ihrer Arbeit benötigte       %%
%% LaTeX-Pakete einbinden.                                                    %%
%% Diese Vorlage kommt bereits mit einigen nützlichen inkludierten Paketen.   %%
%%%%%%%%%%%%%%%%%%%%%%%%%%%%%%%%%%%%%%%%%%%%%%%%%%%%%%%%%%%%%%%%%%%%%%%%%%%%%%%%

%% Macht den \todo-Befehl verfügbar.
%% Hiermit können Sie Abschnitte annotieren,
%% welche weiterer Bearbeitung bedürfen.
\usepackage[textsize=scriptsize]{todonotes}

%% Zeige Zeilennummern in der Arbeit an.
%% Der \linenumbers Befehl muss hierzu aufgerufen werden.
%% Praktisch für Feedback Ihrer potentiellen Korrekturleser!
\usepackage{lineno}
% \linenumbers % <- Kommentar entfernen!


%% Häufig benutzte mathematische Packages.
\usepackage{amsfonts}
\usepackage{amsmath}
\usepackage{amssymb}


\usepackage{listings} % Einbindung von Code
\usepackage{algorithmicx} % Angabe von Algorithmen in Pseudocode
\usepackage{siunitx} % \num Befehl zum einfacheren Formatieren von Zahlen.
\usepackage{enumitem} % Leichter konfigurierbare enumerate-Umgebungen.
\usepackage{subcaption} % Unterteilung von Figures in Subfigures.
\usepackage{hyperref} % Klickbare Referenzen (z.B. im Inhaltsverzeichnis)
\usepackage{url} % \url Kommando für Darstellung von Links
\usepackage{csquotes} % Improved quoting.
\usepackage{xspace} % Nicht terminierte Kommandos essen keinen Whitespace mehr.

%% Tabellen
\usepackage{tabularx} % tabularx Umgebung für mehr Kontrolle über Tabellen.
\usepackage{booktabs} % \toprule, \midrule, \bottomrule
\usepackage{multirow}
\usepackage{multicol}
\usepackage{longtable} % Große Tabellen gehen über mehrere Seiten.

%% Intelligenteres Referenzieren mittels \cref.
%% \languagename um dynamisch zwischen ngerman oder english zu wechseln.
\usepackage[\languagename,capitalize]{cleveref}

%%%%%%%%%%%%%%%%%%%%%%%%%%%%%%%%%%%%%%%%%%%%%%%%%%%%%%%%%%%%%%%%%%%%%%%%%%%%%%%%
%% (Ende) LaTeX Packages in Nutzung                                           %%
%%%%%%%%%%%%%%%%%%%%%%%%%%%%%%%%%%%%%%%%%%%%%%%%%%%%%%%%%%%%%%%%%%%%%%%%%%%%%%%%


\begin{document}
%% Set up title page, declaration of authorship, abstract, acknowledgements
\frontmatter
\makefrontmatter

%%%%%%%%%%%%%%%%%%%%%%%%%%%%%%%%%%%%%%%%%%%%%%%%%%%%%%%%%%%%%%%%%%%%%%%%%%%%%%%%
%% Danksagungen                                                               %%
%%%%%%%%%%%%%%%%%%%%%%%%%%%%%%%%%%%%%%%%%%%%%%%%%%%%%%%%%%%%%%%%%%%%%%%%%%%%%%%%
\begin{acknowledgements}
  Im Falle, dass Sie Ihrer Arbeit eine Danksagung für Ihre Unterstützer
  (Familie, Freunde, Betreuer)
  hinzufügen möchten, können Sie diese hier platzieren.

  Dieser Part ist optional und kann im Quelltext auskommentiert werden.
\end{acknowledgements}
%%%%%%%%%%%%%%%%%%%%%%%%%%%%%%%%%%%%%%%%%%%%%%%%%%%%%%%%%%%%%%%%%%%%%%%%%%%%%%%%
%% (Ende) Danksagungen                                                        %%
%%%%%%%%%%%%%%%%%%%%%%%%%%%%%%%%%%%%%%%%%%%%%%%%%%%%%%%%%%%%%%%%%%%%%%%%%%%%%%%%


\tableofcontents


\mainmatter

%%%%%%%%%%%%%%%%%%%%%%%%%%%%%%%%%%%%%%%%%%%%%%%%%%%%%%%%%%%%%%%%%%%%%%%%%%%%%%%%
%% Der Inhalt der Arbeit                                                      %%
%%                                                                            %%
%% Hier können Sie die schriftliche Ausarbeitung ihrer Arbeit                 %%
%% niederschreiben. Der Übersicht halber bietet sich jedoch an, dies in einer %%
%% oder mehreren separaten Dateien zu tun, welche mittels \input eingebunden  %%
%% werden --- wie auch in der Vorlage geschieht.                              %%
%%%%%%%%%%%%%%%%%%%%%%%%%%%%%%%%%%%%%%%%%%%%%%%%%%%%%%%%%%%%%%%%%%%%%%%%%%%%%%%%

%%%%%%%%%%%%%%%%%%%%%%%%%%%%%%%%%%%%%%%%%%%%%%%%%%%%%%%%%%%%%%%%%%%%%%%%%%%%%%%%
% Diese Datei beinhaltet den eigentlichen Inhalt Ihrer Arbeit.
%
% Es bietet sich der Übersicht halber an, die einzelnen Abschnitte jeweils
% in eigene Dateien zu schreiben und mittels \input einzubinden.
% Eine mögliche Verzeichnisstruktur sähe entsprechend so aus:
%
%     thesis/
%     +- tex/
%     |  +- introduction.tex
%     |  +- motivation.tex
%     |  +- experiments.tex
%     |  |  ...
%     |  +- conclusion.tex
%     +- abstract.tex
%     +- contents.tex
%     +- thesis.tex
%%%%%%%%%%%%%%%%%%%%%%%%%%%%%%%%%%%%%%%%%%%%%%%%%%%%%%%%%%%%%%%%%%%%%%%%%%%%%%%%

\section{Introduction}

\subsection{Motivation}


\subsection{Tools used}

Es wurde der machine learning assistant von intellij verwendet, welcher standartmäßig aktiviert ist, zum schreiben vom Code. Es wurde DeeplWrite/o.Ä. verwendet, um den Text der Arbeit einer grammatik/rechtschreibprüfung zu unterziehen.

\section{Prerequisites}

\subsection{Regular Expressions and Grammars}

Kurze zusammenfassung von regulären ausdrücken und grammatiken

\subsection{Lexer}

Kurze zusammenfassung was ein lexer macht.

\subsection{Parser}

Kurze zusammenfassung was der parser macht.

\subsubsection{LR(1)-Parsing}

Hier nur auf die Tabelle eingehen. Die Erstellung der Tabelle ist für die Arbeit irrelevant, die wird nicht erweiterbar sein und die Generierung ist zu komplex um es sinnvoll in der Arbeit unterzubringen.

\subsubsection{Preprocessor Statements}

Hier kurz darauf eingehen, was preprocessor statements sind.

\section{Implementation}

\subsection{Extendability}

Hier darauf eingehen, in wie weit das Programm erweiterbar ist. Welche Möglichkeiten existieren in den Lexer/Parser einzugreifen?

\subsection{Grammar file syntax}

\subsection{CST-Generation}

Hier darauf eingehen, wie ein Concrete syntax tree vom Parser generiert wird

\subsection{AST-Generation}

Hier darauf eingehen, wie man von einem Concrete syntax tree zu dem abstract syntax tree kommt.

\section{Proof of concept}

\subsection{Java Subset}

Hier kurz darauf eingehen, für welche Sprache eine Grammatik definiert wurde.

\subsection{Automated Refactoring}

Hier Beispiel/Beispiele für automatisierte Refactorings anführen. Beispielsweise Funktionsnamen von underscore zu camelCase überführen.

\section{Conclusion}

Am Ende der Arbeit werden noch einmal die erreichten Ergebnisse
zusammengefasst und diskutiert.


%% Dieser Part kann auskommentiert werden, sollte kein Anhang nötig sein
% \appendix
% \section{Appendix}

Hier können Sie Ihren Anhang definieren.

Achten Sie darauf, dass der Anhang in Ihrer \texttt{thesis.tex}
initial auskommentiert ist.
Der entsprechende Part befindet sich nahe dem Ende der Datei.
Entfernen Sie bei Bedarf die Kommentierung um den Anhang nutzen zu können.


\section{Implemented grammars}

The grammar definitions that are implemented in the \verb|gp-modifiable-ast| library.

\subsection{MiniJava}

\begin{lstlisting}[
  basicstyle=\tiny, %or \small or \footnotesize etc.
]
LANGUAGE_DEF
  name = minijava;
  single_line_comment_available = true;
  single_line_comment_style = //;
  multi_line_comment_available = true;
  multi_line_comment_style_start = /*;
  multi_line_comment_style_end = */;
  case_sensitive = true;
  grammar_start = S;

LEXER_RULES
  whitespace              =   regex(\s+);
  identifier              =   regex([a-zA-Z_][a-zA-Z_0-9]*);
  integer_literal         =   regex([0-9]+(\.[0-9]+)?);
  true_literal            =   "true";
  false_literal           =   "false";
  this_expression         =   regex(this(?!([a-zA-Z_0-9])));
  new_expression          =   regex(new(?!([a-zA-Z_0-9])));
  class_stmt              =   regex(class(?!([a-zA-Z_0-9])));
  public_stmt             =   regex(public(?!([a-zA-Z_0-9])));
  static_stmt             =   regex(static(?!([a-zA-Z_0-9])));
  return_stmt             =   regex(return(?!([a-zA-Z_0-9])));
  extents_stmt            =   regex(extends(?!([a-zA-Z_0-9])));
  void_type               =   regex(void(?!([a-zA-Z_0-9])));
  main_identifier         =   regex(main(?!([a-zA-Z_0-9])));
  string_type             =   regex(String(?!([a-zA-Z_0-9])));
  int_type                =   regex(int(?!([a-zA-Z_0-9])));
  boolean_type            =   regex(boolean(?!([a-zA-Z_0-9])));
  array_length_stmt       =   regex(length(?!([a-zA-Z_0-9])));
  not                     =   "!";
  bracket_open            =   "(";
  bracket_close           =   ")";
  curly_bracket_open      =   "{";
  curly_bracket_close     =   "}";
  square_bracket_open     =   "[";
  square_bracket_close    =   "]";
  dot                     =   ".";
  comma                   =   ",";
  semicolon               =   ";";
  if                      =   regex(if(?!([a-zA-Z_0-9])));
  else                    =   regex(else(?!([a-zA-Z_0-9])));
  while                   =   regex(while(?!([a-zA-Z_0-9])));
  assignment              =   "=";
  plus                    =   "+";
  minus                   =   "-";
  multiply                =   "*";
  less_then               =   "<";
  and                     =   "&&";
  or                      =   "||";
  println                 =   "System.out.println";
  single_line_comment     =   customMatcher(singleLineCommentMatcher);
  multi_line_comment      =   customMatcher(multiLineCommentMatcher);

HIDDEN_LEXER_RULES
  whitespace, single_line_comment, multi_line_comment;

PRODUCTIONS
  S                           ->  MAIN_CLASS CLASS_DECLARATIONS[inline];
  CLASS_DECLARATIONS[list]    ->  CLASS_DECLARATION CLASS_DECLARATIONS | EPSILON;
  CLASS_DECLARATION           ->  class_stmt[hidden] identifier[alias=className] curly_bracket_open[hidden] 
                                  VAR_DECLARATIONS[inline] METHOD_DECLARATIONS[inline] curly_bracket_close[hidden];
  CLASS_DECLARATION           ->  class_stmt[hidden] identifier[alias=className] extents_stmt[hidden] 
                                  identifier[alias=extendedClassName] curly_bracket_open[hidden] 
                                  VAR_DECLARATIONS[inline] METHOD_DECLARATIONS[inline] curly_bracket_close[hidden];
  MAIN_CLASS                  ->  class_stmt[hidden] identifier[alias=className] curly_bracket_open[hidden] 
                                  public_stmt static_stmt void_type main_identifier bracket_open[hidden] string_type 
                                  square_bracket_open square_bracket_close identifier[alias=varName] 
                                  bracket_close[hidden] curly_bracket_open[hidden] VAR_DECLARATIONS[inline] 
                                  STATEMENTS[inline] curly_bracket_close[hidden] curly_bracket_close[hidden];
  METHOD_DECLARATIONS[list]   ->  METHOD_DECLARATION METHOD_DECLARATIONS | EPSILON;
  METHOD_DECLARATION          ->  public_stmt TYPE identifier[alias=functionName] bracket_open[hidden] 
                                  FORMAL_PARAMETER_LIST[inline] bracket_close[hidden] curly_bracket_open[hidden] 
                                  VAR_DECLARATIONS[inline] STATEMENTS[inline] return_stmt EXPRESSION 
                                  semicolon[hidden] curly_bracket_close[hidden];
  FORMAL_PARAMETER_LIST[list] ->  FORMAL_PARAMETER_LIST comma[hidden] FORMAL_PARAMETER | FORMAL_PARAMETER | EPSILON;
  FORMAL_PARAMETER            ->  TYPE identifier[alias=parameterName];
  VAR_DECLARATIONS[list]      ->  VAR_DECLARATIONS VAR_DECLARATION | EPSILON;
  STATEMENTS[list]            ->  STATEMENT STATEMENTS | EPSILON;
  VAR_DECLARATION             ->  TYPE identifier[alias=varName] semicolon[hidden];
  STATEMENT                   ->  BLOCK | 
                                  ASSIGNMENT_STATEMENT | 
                                  ARRAY_ASSIGNMENT_STATEMENT | 
                                  IF_STATEMENT | 
                                  WHILE_STATEMENT | 
                                  PRINT_STATEMENT;
  BLOCK                       ->  curly_bracket_open[hidden] STATEMENTS[inline] curly_bracket_close[hidden];
  ASSIGNMENT_STATEMENT        ->  identifier[alias=varName] assignment[hidden] EXPRESSION semicolon[hidden];
  ARRAY_ASSIGNMENT_STATEMENT  ->  identifier[alias=varName] square_bracket_open[hidden] EXPRESSION 
                                  square_bracket_close[hidden] assignment[hidden] EXPRESSION semicolon[hidden];
  IF_STATEMENT                ->  if[hidden] bracket_open[hidden] EXPRESSION bracket_close[hidden] 
                                  STATEMENT[alias=then] else[hidden] STATEMENT[alias=else];
  WHILE_STATEMENT             ->  while[hidden] bracket_open[hidden] EXPRESSION bracket_close[hidden] STATEMENT;
  PRINT_STATEMENT             ->  println[hidden] bracket_open[hidden] EXPRESSION bracket_close[hidden] 
                                  semicolon[hidden];
  TYPE                        ->  ARRAY_TYPE | 
                                  BOOLEAN_TYPE | 
                                  INTEGER_TYPE | 
                                  identifier[alias=className];
  ARRAY_TYPE                  ->  int_type square_bracket_open[hidden] square_bracket_close[hidden];
  BOOLEAN_TYPE                ->  boolean_type;
  INTEGER_TYPE                ->  int_type;
  EXPRESSION                  ->  AND_EXPRESSION
                                  | COMPARE_EXPRESSION
                                  | PLUS_EXPRESSION
                                  | MINUS_EXPRESSION
                                  | TIMES_EXPRESSION
                                  | ARRAY_LOOKUP
                                  | ARRAY_LENGTH
                                  | MESSAGE_SEND
                                  | CLAUSE[inline];
  AND_EXPRESSION              ->  CLAUSE[inline,alias=left] and[hidden] CLAUSE[inline,alias=right];
  COMPARE_EXPRESSION          ->  PRIMARY_EXPRESSION[inline,alias=left] less_then[hidden] 
                                  PRIMARY_EXPRESSION[inline,alias=right];
  PLUS_EXPRESSION             ->  PRIMARY_EXPRESSION[inline,alias=left] plus[hidden] 
                                  PRIMARY_EXPRESSION[inline,alias=right];
  MINUS_EXPRESSION            ->  PRIMARY_EXPRESSION[inline,alias=left] minus[hidden] 
                                  PRIMARY_EXPRESSION[inline,alias=right];
  TIMES_EXPRESSION            ->  PRIMARY_EXPRESSION[inline,alias=left] multiply[hidden] 
                                  PRIMARY_EXPRESSION[inline,alias=right];
  ARRAY_LOOKUP                ->  PRIMARY_EXPRESSION[inline] square_bracket_open[hidden] PRIMARY_EXPRESSION[inline] square_bracket_close[hidden];
  ARRAY_LENGTH                ->  PRIMARY_EXPRESSION[inline] dot[hidden] array_length_stmt;
  MESSAGE_SEND                ->  PRIMARY_EXPRESSION[inline,alias=calledOn] dot[hidden] identifier[alias=functionName] bracket_open[hidden] bracket_close[hidden] |
                                  PRIMARY_EXPRESSION[inline,alias=calledOn] dot[hidden] identifier[alias=functionName] bracket_open[hidden] EXPRESSION_LIST bracket_close[hidden];
  EXPRESSION_LIST[list]       ->  EXPRESSION | EXPRESSION_LIST comma[hidden] EXPRESSION;
  CLAUSE                      ->  NOT_EXPRESSION | PRIMARY_EXPRESSION[inline];
  NOT_EXPRESSION              ->  not[hidden] CLAUSE[inline];
  PRIMARY_EXPRESSION          ->  integer_literal
                                  | true_literal
                                  | false_literal
                                  | identifier[alias=varName]
                                  | this_expression
                                  | ARRAY_ALLOCATION_EXPRESSION
                                  | ALLOCATION_EXPRESSION
                                  | BRACKET_EXPRESSION;
  ARRAY_ALLOCATION_EXPRESSION ->  new_expression[hidden] int_type square_bracket_open[hidden] EXPRESSION square_bracket_close[hidden];
  ALLOCATION_EXPRESSION       ->  new_expression[hidden] identifier[alias=className] bracket_open[hidden] bracket_close[hidden];
  BRACKET_EXPRESSION          ->  bracket_open[hidden] EXPRESSION bracket_close[hidden];
\end{lstlisting}

  \subsection{Unterabschnitte}
  Die Verwendung von Unterabschnitten im Anhang
  mittels \texttt{\textbackslash subsection}
  funktioniert ebenfalls!


%%%%%%%%%%%%%%%%%%%%%%%%%%%%%%%%%%%%%%%%%%%%%%%%%%%%%%%%%%%%%%%%%%%%%%%%%%%%%%%%
%% (Ende) Der Inhalt der Arbeit                                               %%
%%%%%%%%%%%%%%%%%%%%%%%%%%%%%%%%%%%%%%%%%%%%%%%%%%%%%%%%%%%%%%%%%%%%%%%%%%%%%%%%


\backmatter
\listoffigures
\listoftables

\clearpage
\bibliography{references}
%% Depending on Language, use german alphadin or original alpha
\iflanguage{ngerman}{
  \bibliographystyle{alphadin}
}{
  \bibliographystyle{alpha}
}

\end{document}
