
% Basic introduction
Software developers are often facing the issue of maintaining large amount of source code. 
The keep the source code up to date, often times a process called refactoring is needed, in which the software developers update the code.

% Detailed background
To allow the software developers to perform these updates automatically, most integrated development environments (IDE) provide tools to 
perform predefined refactorings automatically. 
The effort of performing some refactorings on large code bases can be too much to perform manually. 
Most of the times, the tools provided by the IDE are sufficient to perform the refactorings, but sometimes a custom solution is required.
One way to create automated custom refactorings is by parsing the source code and building it into an abstract syntax tree (AST), 
which can be maniup

% General problem
The tools given by an IDE or other sources are sometimes not sufficient to perform an automated refactoring, 
therefore other methods of generating custom refactorings are required. 

% Summarizing result
Here we provide a Java library, which can parse a subset of the context-free language based on their corresponding grammar definition 
and generates a AST which can be modified by the user and afterwards reverted back to the source code.

% What does the main result reveal?

The program provides an relatively easy method of safe large scale refactorings, however the effort of defining a grammar is large.
It also has been shown, that this tool can also be used to extend a given programming language and allow for additional features to be added, 
basicly allowing for an primitive form of source to source compilation. 
We can also see clear limits with this tool, as it does not include features that are specific to each programming language, like symbol tables, 
which are not defined by the grammar of the language.
This limits the amount of refactorings that can be archived without implementing additional functionality on top of the base program.